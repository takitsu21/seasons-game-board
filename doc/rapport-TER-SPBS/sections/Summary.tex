\part{Summary} 

In the end, this project was one of the most difficult we've ever had. Not because it was particularly difficult but because of how time-consuming and dense it was. The game contains an enormous amount of elements, each one having its own particularity and therefore its load of issues. Putting them all together correctly was already a challenge, and we had to add the Strategy pattern which required a lot of tedious coding.\\

On top of all this, the TER project required a lot of time and reflection because of all the possibilities we had to take in count when creating an AI. While we had first planned to simply implement a Monte-Carlo AI, we hadn't anticipated that being four for that work would've been too much, so we had to improvise another type of AI. In the meantime, the other projects prevented us from focusing on the project as much as we would have wanted to.\\

Despite all those obstacles, we still managed to implement an efficient Monte-Carlo algorithm which wins against most of the basic AIs, and we came up with a simpler AI that has a great potential. However, in spite of our efforts to create an ambitious AI as powerful as possible, the best ones remain the composed AI "Compose Dyl" and "Compose Marg". All these different types of player are working together on the game we created in the previous part of the project.

In fine, this project taught us how complicated it can be to add a sophisticated AI on a game without anticipating it during the development of the said game. Still, we learnt to use new patterns we had never encountered before, and we discovered the Monte-Carlo algorithm which we were quite curious about for a while.